\documentclass[../main.tex]{subfiles}
\begin{document}
\begin{tcolorbox}[title=Opgave 1,
    colback=blue!1!white,
    colframe=black,
    colbacktitle=blue!25!white,
    coltitle=red!25!black,
    fonttitle=\bfseries,
    subtitle style={boxrule=0.4pt,
    colback=blue!7!white} ]
    \tcbsubtitle{1a}
        Højden \(h\) kan bestemmes ved at indsætte \(t=0\) i vektorfunktionen.\\
        Afstanden \(x_{\text{max}}\) kan bestemmes ved at løse ligningen \(50-5t^2=0\) for \(t \geq 0\), da højden ved tårnet fod er 0.
    \tcbsubtitle{1b}
        Vinklen mellem positionsvektoren \(\vec{r}(t)\) og hastighedsvektoren \(\vec{v}(t)\) for \(t = 2\) kan bestemmes ved:
        \[v = \text{arccos} \left( \frac{\vec{r}(2) \bullet \vec{v}(2)}{|\vec{r}(2)| \cdot |\vec{v}(2)|}\right)\]
    \tcbsubtitle{1c}
        Korteste afstand til tårnets fod kan bestemmes ved hjælp af en distancefunktion. Afstanden fra
        et givent punkt til tårnets fod kan skrives ved:
        \[\text{dist}(t) = \sqrt{(8t)^2+(50-5t^2)^2}\]
        \(t\)-værdien for korteste afstand må være løsningen til ligningen:
        \[\text{dist}'(t)=0\]
        Den korteste afstand findes ved at indsætte \(t\) i distancefunktionen.
\end{tcolorbox}
\end{document}