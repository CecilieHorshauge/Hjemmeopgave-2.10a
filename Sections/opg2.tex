\documentclass[../main.tex]{subfiles}
\begin{document}
\begin{tcolorbox}[title=Opgave 2,
    colback=blue!1!white,
    colframe=black,
    colbacktitle=blue!25!white,
    coltitle=red!25!black,
    fonttitle=\bfseries,
    subtitle style={boxrule=0.4pt,
    colback=blue!7!white} ]
    \tcbsubtitle{2a}
        Banekurven for \(M\) skærer x-aksen netop når y-værdien er 0.\\
        Man løser ligningen: 
        \[1.5-t^2=0\]
        Derefter indsættes \(t\) i \(0.5 \cdot t\), dette er x-koordinaten for banekurvens skæringspunkt med x-aksen.
    \tcbsubtitle{2b}
        Farten for vogn \(M\) når \(t = 0.5\) kan regnes ved:
        \[\text{fart}(t)=\left| \overrightarrow{v}(t)\right|= \sqrt{x'(t)^2+y'(t)^2}\]
        Vær opmærksom på at enheden bliver i km/min. Konverter dette til km/t.
    \tcbsubtitle{2c}
        For at vognene kan støde sammen må der findes et tidspunkt de har samme koordinatsæt.
        Man kan løse to ligninger med 2 ubekendte:
        \begin{align}
            0.5t&=t\cdot \text{cos}(60)\\
            1.5-t^2&=t\cdot \text{sin}(60)
        \end{align}
        Dette er tiden for sammenstødet. Hvis der ingen løsning er for ligningssystemet støder de aldrig sammen.
\end{tcolorbox}
\end{document}