\documentclass[../main.tex]{subfiles}
\begin{document}
\begin{tcolorbox}[title=Opgave 4,
    colback=blue!1!white,
    colframe=black,
    colbacktitle=blue!25!white,
    coltitle=red!25!black,
    fonttitle=\bfseries,
    subtitle style={boxrule=0.4pt,
    colback=blue!7!white} ]
    \tcbsubtitle{4a}
        Løs ligningen \(y(t)=0\) og sortér i outputtet. Korrekt \(t\)-værdi indsættes i \(x(t)\).\\
        Svar angives i meter.
    \tcbsubtitle{4b}
        Først bestemmes tidspunktet, hvor kassen kommer højest op. Det gør man ved at løse ligningen \(y(t)'=0\).\\
        En hastighedsvektor bergnes ved:
        \[\vec{v}(t)=\vec{r}\,'(t)=\begin{pmatrix} x'(t) \\ y'(t) \end{pmatrix}\]
        Den fundne tid \(t\) indsættes.
    \tcbsubtitle{4c}
        \(t=0\)\\ 
        Accelerationsvektoren kan bestemmes ved:
        \[\vec{a}(t)=\vec{v}\,'(t) = \vec{r}\, '' (t)=\begin{pmatrix} x''(t) \\ y''(t) \end{pmatrix}\]
        Koordinaterne omdannes til en retningsvinkel.\\
        Størrelsen af accelerationsvektoren beregnes ved:
        \[acc(t)=|\vec{a}(t)|= \sqrt{x''(t)^2+y''(t)^2}\]
\end{tcolorbox}
\end{document}