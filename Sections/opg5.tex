\documentclass[../main.tex]{subfiles}
\begin{document}
\begin{tcolorbox}[title=Opgave 5,
    colback=blue!1!white,
    colframe=black,
    colbacktitle=blue!25!white,
    coltitle=red!25!black,
    fonttitle=\bfseries,
    subtitle style={boxrule=0.4pt,
    colback=blue!7!white} ]
    \tcbsubtitle{5a}
        Løs ligningen \(y(t)=0\) ,  \(t\)-værdierne er tidspunkterne, hvor banekurven skærer x-aksen.
    \tcbsubtitle{5b}
        En hastighedsvektor bestemmes ved:
        \[\vec{a}(t)=\vec{v}\,'(t) = \vec{r}\, '' (t)=\begin{pmatrix} x''(t) \\ y''(t) \end{pmatrix}\]
    \tcbsubtitle{5c}
        \(t = 10\)\\
        Længden af accelerationsvektoren bestemmes ved:
        \[acc(t)=|\vec{a}(t)|= \sqrt{x''(t)^2+y''(t)^2}\]
\end{tcolorbox}
\vspace*{2 cm}
\end{document}