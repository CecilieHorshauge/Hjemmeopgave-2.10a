\documentclass[../main.tex]{subfiles}
\begin{document}
\begin{tcolorbox}[title=Opgave 6,
    colback=blue!1!white,
    colframe=black,
    colbacktitle=blue!25!white,
    coltitle=red!25!black,
    fonttitle=\bfseries,
    subtitle style={boxrule=0.4pt,
    colback=blue!7!white} ]
    \tcbsubtitle{6a}
        Man indsætter x- og y-koordinat på x og y's plads i ligningen og tjekker om der er lighed.
    \tcbsubtitle{6b}
        Tagentens ligning skrives ved:
        \[y=f'(x_0)\cdot (x-x_0)+f(x_0)\]
        For at finde hældningskoefficienten benyttes implicit differentiation og indsættes på \(f'(x_0)\)'s plads.\\
        Vi kender allerede resten:\;\; \(x_0=2\)\;\;\; \(f(x_0)=-1\)
\end{tcolorbox}
\end{document}